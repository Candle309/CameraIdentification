\begin{abstract}

L'obiettivo di questo elaborato è di implementare un sistema di identificazione di camera (Source Camera Identification, SCI) basato sul clustering di immagini in gruppi eterogenei, utilizzando come distanza la PCE, peaks-to-correlation energy ratio, tra i PRNU delle immagini. L'algoritmo di clustering implementato è Normalized Cuts.
Una volta creati i gruppi di immagini, per ciascuno di essi verrà estratta la PRNU di camera e sarà dunque possibile comparare una nuova immagine per stabilire a quale camera appartiene.
Per validare il metodo implementato sono proposti diversi risultati sperimentali basati su dataset differenti. 
Ulteriore analisi è stata effettuata utilizzando immagini caricate sui social network, in particolare con immagini provenienti da Facebook.
\end{abstract}

