\begin{tframe}{Conclusioni}

Il metodo per l'identificazione automatica di dispositivi implementato utilizza la PRNU per descrivere le immagini e la PCE come misura di similarità fra PRNU. L'algoritmo Normalized Cuts partiziona le immagini in clusters. Tali clusters rappresentano gruppi di immagini simili a partire dalle quali le fingerprints di camera verranno calcolate.

\vspace{0.1in}

La fase di validazione ha permesso di trarre alcune conclusioni:
\begin{itemize}
\item Per quanto riguarda le immagini naturali, i risultati ottenuti sono ottimi. Infatti il metodo è perfettamente in grado di associare una generica immagine alla fingerprint di camera corretta.
\item Il caso delle immagini scaricate da Facebook presenta risultati altalenanti. La compressione subita dalle immagini incide negativamente sull'accuratezza della predizione, a causa di una minore qualità delle PRNU estratte che si riflette sulla PCE.
\end{itemize}

\end{tframe}