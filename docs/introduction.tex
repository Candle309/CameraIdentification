\section{Introduzione}

Con l’enorme diffusione di dispositivi in grado di acquisire ed elaborare
immagini digitali, identificare in modo affidabile i sensori a partire da una
immagine diventa sempre più una necessità. Questo è vero specialmente in
ambito forense, per determinare l'origine di scatti presentati come prova in
un’aula di tribunale. Metodi di identificazione, inoltre, potrebbero essere utili
per dimostrare che alcune immagini sono state ottenute utilizzando una
specifica fotocamera piuttosto che elaborate al PC tramite programmi ad hoc.
Nei successivi paragrafi analizzeremo il problema dell’identificazione del
sensore, tramite l’uso del rumore delle immagini che in un certo senso
identifica univocamente la fonte di acquisizione . 


Il problema dell'identificazione della fotocamera digitale da cui una foto è stata scattata è un problema noto dell'\emph{image forensics}. 

Descrizione generale del problema: citare articolo, l'implementazione parte da quello.

Come funziona: step dell'algoritmo

1. estrazione prnu e calcolo pce
2. clustering 
3. fingerprint
4. validazione risultati

(citare relative sezioni)

Linguaggio usato