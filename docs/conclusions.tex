\section{Conclusioni}

In questo articolo abbiamo presentato un metodo di identificazione automatica di dispositivi a partire da un insieme di immagini la cui provenienza è ignota. Il metodo, prendendo le basi da \cite{ Amerini2014831} e apportandone alcune modifiche, funziona come segue:
\begin{enumerate}
\item Le PRNU vengono estratte dalle immagini del dataset. 
\item Per ogni coppia di PRNU viene calcolata una misura di similarità utilizzando la PCE e viene costruita una matrice dei pesi che rappresenta il grafo completamente connesso.
\item L'algoritmo Normalized Cuts partiziona il grafo e divide le immagini in clusters.
\item Per ogni clusters viene calcolata una fingerprint che rappresenta un dispositivo, a partire dalle PRNU delle immagini appartenenti a tale cluster.
\item Infine una fase di validazione che verifica la qualità delle fingerprint estratte.
\end{enumerate}

Il dataset utilizzato contiene immagini direttamente acquisite dai dispositivi e le stesse immagini caricate e poi scaricate da Facebook. I risultati ottenuti nella fase di validazione differiscono a seconda del tipo di immagini utilizzate. Le fingerprint estratte per il caso delle immagini naturali sono di qualità ed il metodo è in grado di associare una generica immagine del dataset alla fingerprint corretta, ad eccezione per le immagini provenienti dal dispositivo Huawei.

Per quanto riguarda le immagini scaricate da Facebook, il metodo si è mostrato meno efficace. Questo è dovuto alla fatto che le immagini caricate su un social network subiscono, in genere, una compressione che va peggiorare la qualità delle PRNU estratte. Questo ha conseguenze sulla misura di similarità, in questo caso la PCE. La similarità fra immagini provenienti dallo stesso dispositivo risulta meno forte rispetto a considerare la similarità delle stesse immagini per il caso delle immagini naturali. Inoltre a causa di ciò, la fase di clustering produce un numero maggior di clusters che, seppur puliti, conterranno un numero minore di immagini rispetto al caso delle immagini naturali, a causa di una TPR molto più bassa; tutto ciò va ad influire sulla qualità delle fingerprints nella fase di validazione. Va notato comunque, che la compressione, perlomeno quella effettuata da Facebook, sembra incidere maggiormente su alcuni dispositivi piuttosto che su altri, con alcuni di questi che infatti ottengono un'accuratezza nella predizione alta indipendentemente dalla ripartizione del dataset in cui si ritrovano, come mostrato nelle matrici di confusione dei vari esperimenti.

In conclusione, il metodo implementato funziona correttamente quando si estraggono delle PRNU di qualità dalle immagini, ovvero pulite e calcolate con tante immagini, cosa meno possibile se l'immagine è stata sottoposta ad una pesante compressione.